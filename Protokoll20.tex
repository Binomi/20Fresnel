% Für Bindekorrektur als optionales Argument "BCORfaktormitmaßeinheit", dann
% sieht auch Option "twoside" vernünftig aus
% Näheres zu "scrartcl" bzw. "scrreprt" und "scrbook" siehe KOMA-Skript Doku
\documentclass[12pt,a4paper,titlepage,headinclude,bibtotoc]{scrartcl}


%---- Allgemeine Layout Einstellungen ------------------------------------------

% Für Kopf und Fußzeilen, siehe auch KOMA-Skript Doku
\usepackage[komastyle]{scrpage2}
\pagestyle{scrheadings}
\automark[section]{chapter}
\setheadsepline{0.5pt}[\color{black}]

%keine Einrückung
\parindent0pt

%Einstellungen für Figuren- und Tabellenbeschriftungen
\setkomafont{captionlabel}{\sffamily\bfseries}
\setcapindent{0em}

\usepackage{caption}

%---- Weitere Pakete -----------------------------------------------------------
% Die Pakete sind alle in der TeX Live Distribution enthalten. Wichtige Adressen
% www.ctan.org, www.dante.de

% Sprachunterstützung
\usepackage[ngerman]{babel}

% Benutzung von Umlauten direkt im Text
% entweder "latin1" oder "utf8"
\usepackage[utf8]{inputenc}

% Pakete mit Mathesymbolen und zur Beseitigung von Schwächen der Mathe-Umgebung
\usepackage{latexsym,exscale,amssymb,amsmath}

% Weitere Symbole
\usepackage[nointegrals]{wasysym}
\usepackage{eurosym}

% Anderes Literaturverzeichnisformat
%\usepackage[square,sort&compress]{natbib}

% Für Farbe
\usepackage{color}

% Zur Graphikausgabe
%Beipiel: \includegraphics[width=\textwidth]{grafik.png}
\usepackage{graphicx}

% Text umfließt Graphiken und Tabellen
% Beispiel:
% \begin{wrapfigure}[Zeilenanzahl]{"l" oder "r"}{breite}
%   \centering
%   \includegraphics[width=...]{grafik}
%   \caption{Beschriftung} 
%   \label{fig:grafik}
% \end{wrapfigure}
\usepackage{wrapfig}

% Mehrere Abbildungen nebeneinander
% Beispiel:
% \begin{figure}[htb]
%   \centering
%   \subfigure[Beschriftung 1\label{fig:label1}]
%   {\includegraphics[width=0.49\textwidth]{grafik1}}
%   \hfill
%   \subfigure[Beschriftung 2\label{fig:label2}]
%   {\includegraphics[width=0.49\textwidth]{grafik2}}
%   \caption{Beschriftung allgemein}
%   \label{fig:label-gesamt}
% \end{figure}
\usepackage{subfigure}
\usepackage{adjustbox}

% Caption neben Abbildung
% Beispiel:
% \sidecaptionvpos{figure}{"c" oder "t" oder "b"}
% \begin{SCfigure}[rel. Breite (normalerweise = 1)][hbt]
%   \centering
%   \includegraphics[width=0.5\textwidth]{grafik.png}
%   \caption{Beschreibung}
%   \label{fig:}
% \end{SCfigure}
\usepackage{sidecap}

% Befehl für "Entspricht"-Zeichen
\newcommand{\corresponds}{\ensuremath{\mathrel{\widehat{=}}}}

%Für chemische Formeln (von www.dante.de)
%% Anpassung an LaTeX(2e) von Bernd Raichle
\makeatletter
\DeclareRobustCommand{\chemical}[1]{%
  {\(\m@th
   \edef\resetfontdimens{\noexpand\)%
       \fontdimen16\textfont2=\the\fontdimen16\textfont2
       \fontdimen17\textfont2=\the\fontdimen17\textfont2\relax}%
   \fontdimen16\textfont2=2.7pt \fontdimen17\textfont2=2.7pt
   \mathrm{#1}%
   \resetfontdimens}}
\makeatother

%Si Einheiten
\usepackage{siunitx}

%c++ Code einbinden
\usepackage{listings}
\lstset{numbers=left, numberstyle=\tiny, numbersep=5pt}

%Differential
\newcommand{\dif}{\ensuremath{\mathrm{d}}}

%Boxen,etc.
\usepackage{fancybox}
\usepackage{empheq}

%Fußnoten auf gleiche Seite
\interfootnotelinepenalty=1000

%Dateien aus Unterverzeichnissen
\usepackage{import}

%Bibliography \bibliography{literatur} und \cite{gerthsen}
%\usepackage{cite}
\usepackage{babelbib}
\selectbiblanguage{ngerman}

\begin{document}

\begin{titlepage}
\centering
\textsc{\Large Anfängerpraktikum der Fakultät für
  Physik,\\[1.5ex] Universität Göttingen}

\vspace*{4.2cm}

\rule{\textwidth}{1pt}\\[0.5cm]
{\huge \bfseries
  Fresnelsche Formeln und\\[1.5ex]
  und Polarisation}\\[0.5cm]
\rule{\textwidth}{1pt}

\vspace*{3.0cm}

\begin{Large}
\begin{tabular}{ll}
Praktikant:
 	&  Felix Kurtz\\
Versuchspartner:
 	&  Michael Lohmann\\

E-Mail: 
	&  felix.kurtz@stud.uni-goettingen.de\\
	
 Betreuer: & Phillip Bastian\\
 Versuchsdatum: &  06.03.2015\\
\end{tabular}
\end{Large}

\vspace*{1.8cm}

\begin{Large}
\fbox{
  \begin{minipage}[t][2.5cm][t]{6cm} 
    Eingegangen am:
  \end{minipage}
}
\end{Large}

\end{titlepage}

\tableofcontents

\newpage

\section{Einleitung}
\label{sec:einleitung}

\section{Theorie}
\label{sec:theorie}
\subsection{Fresnelsche Formeln}
\begin{align}
	r_s&=-\frac{\sin(\alpha-\beta)}{\sin(\alpha+\beta)}\\
	r_p&=\frac{\tan(\alpha-\beta)}{\tan(\alpha+\beta)}
\end{align}

\begin{figure}[!h]
	\centering
	\includegraphics[scale=0.7]{fresnelkoeff.png}
	\caption{Fresnelkoeffizienten für $n=1.51$. \cite[Datum: 23.03.2015]{LP20}}
	\label{fig:fresnelkoeff}
\end{figure}

\subsection{Brewster-Winkel}
\begin{align}
	\tan\alpha_\text{Brewster}=\frac{n_2}{n_1}
\end{align}

\section{Durchführung}
\label{sec:durchfuehrung}
\begin{figure}[!h]
	\centering
	\includegraphics[scale=0.7]{aufbau_schema.png}
	\caption{Versuchsaufbau schematisch. \cite[Datum: 23.03.2015]{LP20}}
	\label{fig:aufbau}
\end{figure}
\begin{figure}[!h]
	\centering
	\includegraphics[scale=0.7]{nicol.png}
	\caption{Strahlengang im Nicolschen Prisma. \cite[Datum: 23.03.2015]{LP20}}
	\label{fig:nicol}
\end{figure}



\section{Auswertung}
\label{sec:auswertung}
\subsection{Drehung}
\begin{figure}[!htb]
	\centering
	\input{drehung.tex}
	\caption{Drehwinkel $\gamma$ gegen den Auftreffwinkel $\alpha$ aufgetragen.}
\end{figure}

\begin{align}
	\sigma_n=\frac{\sigma_\alpha}{\cos^2\alpha}
\end{align}

Aus der linearen Regression: $\alpha=53.5^\circ \pm 0.1^\circ$.
\begin{empheq}[box=\shadowbox]{align*}
	n = 1.352 \pm 0.005\,.
\end{empheq}

Aus dem $\chi^2$-Fit der Theoriekurve erhält man
\begin{empheq}[box=\shadowbox]{align*}
	n = 1.405 \pm 0.019\,.
\end{empheq}

\subsection{Brewsterwinkel}
\begin{table}[!htb]
	\centering
	\begin{tabular}{|c|c|c|c|}
		\hline		
		& $\Phi$ [$^\circ$] & $\alpha$ [$^\circ$] & $n$ \\
		\hline
		Alle Werte & $66.6 \pm 0.6$ & $56.7 \pm 0.3$ & $1.522 \pm 0.018$ \\
		erste Messreihe & $66.4 \pm 0.9$ & $56.8 \pm 0.5$ & $1.53 \pm 0.03$ \\
		zweite Messreihe & $67.0 \pm 0.5$ & $56.50 \pm 0.25$ & $1.511 \pm 0.015$ \\
		\hline
		Michael & $67.3 \pm 0.5$ & $56.35 \pm 0.25$ & $1.502 \pm 0.015$ \\
		Felix & $65.9 \pm 1.0$ & $57.1 \pm 0.5$ & $1.55 \pm 0.03$ \\
		\hline
	\end{tabular}
	\caption{Brewsterwinkel.}
	\label{tab:brewster}
\end{table}

\section{Diskussion}
\label{sec:diskussion}

\section{Anhang}

\bibliography{literatur}
\bibliographystyle{babalpha}

\end{document}
