% Für Bindekorrektur als optionales Argument "BCORfaktormitmaßeinheit", dann
% sieht auch Option "twoside" vernünftig aus
% Näheres zu "scrartcl" bzw. "scrreprt" und "scrbook" siehe KOMA-Skript Doku
\documentclass[12pt,a4paper,titlepage,headinclude,bibtotoc]{scrartcl}


%---- Allgemeine Layout Einstellungen ------------------------------------------

% Für Kopf und Fußzeilen, siehe auch KOMA-Skript Doku
\usepackage[komastyle]{scrpage2}
\pagestyle{scrheadings}
\automark[section]{chapter}
\setheadsepline{0.5pt}[\color{black}]

%keine Einrückung
\parindent0pt

%Einstellungen für Figuren- und Tabellenbeschriftungen
\setkomafont{captionlabel}{\sffamily\bfseries}
\setcapindent{0em}

\usepackage{caption}

%---- Weitere Pakete -----------------------------------------------------------
% Die Pakete sind alle in der TeX Live Distribution enthalten. Wichtige Adressen
% www.ctan.org, www.dante.de

% Sprachunterstützung
\usepackage[ngerman]{babel}

% Benutzung von Umlauten direkt im Text
% entweder "latin1" oder "utf8"
\usepackage[utf8]{inputenc}

% Pakete mit Mathesymbolen und zur Beseitigung von Schwächen der Mathe-Umgebung
\usepackage{latexsym,exscale,amssymb,amsmath}

% Weitere Symbole
\usepackage[nointegrals]{wasysym}
\usepackage{eurosym}

% Anderes Literaturverzeichnisformat
%\usepackage[square,sort&compress]{natbib}

% Für Farbe
\usepackage{color}

% Zur Graphikausgabe
%Beipiel: \includegraphics[width=\textwidth]{grafik.png}
\usepackage{graphicx}

% Text umfließt Graphiken und Tabellen
% Beispiel:
% \begin{wrapfigure}[Zeilenanzahl]{"l" oder "r"}{breite}
%   \centering
%   \includegraphics[width=...]{grafik}
%   \caption{Beschriftung} 
%   \label{fig:grafik}
% \end{wrapfigure}
\usepackage{wrapfig}

% Mehrere Abbildungen nebeneinander
% Beispiel:
% \begin{figure}[htb]
%   \centering
%   \subfigure[Beschriftung 1\label{fig:label1}]
%   {\includegraphics[width=0.49\textwidth]{grafik1}}
%   \hfill
%   \subfigure[Beschriftung 2\label{fig:label2}]
%   {\includegraphics[width=0.49\textwidth]{grafik2}}
%   \caption{Beschriftung allgemein}
%   \label{fig:label-gesamt}
% \end{figure}
\usepackage{subfigure}
\usepackage{adjustbox}

% Caption neben Abbildung
% Beispiel:
% \sidecaptionvpos{figure}{"c" oder "t" oder "b"}
% \begin{SCfigure}[rel. Breite (normalerweise = 1)][hbt]
%   \centering
%   \includegraphics[width=0.5\textwidth]{grafik.png}
%   \caption{Beschreibung}
%   \label{fig:}
% \end{SCfigure}
\usepackage{sidecap}

% Befehl für "Entspricht"-Zeichen
\newcommand{\corresponds}{\ensuremath{\mathrel{\widehat{=}}}}

%Für chemische Formeln (von www.dante.de)
%% Anpassung an LaTeX(2e) von Bernd Raichle
\makeatletter
\DeclareRobustCommand{\chemical}[1]{%
  {\(\m@th
   \edef\resetfontdimens{\noexpand\)%
       \fontdimen16\textfont2=\the\fontdimen16\textfont2
       \fontdimen17\textfont2=\the\fontdimen17\textfont2\relax}%
   \fontdimen16\textfont2=2.7pt \fontdimen17\textfont2=2.7pt
   \mathrm{#1}%
   \resetfontdimens}}
\makeatother

%Si Einheiten
\usepackage{siunitx}

%c++ Code einbinden
\usepackage{listings}
\lstset{numbers=left, numberstyle=\tiny, numbersep=5pt}

%Differential
\newcommand{\dif}{\ensuremath{\mathrm{d}}}

%Boxen,etc.
\usepackage{fancybox}
\usepackage{empheq}

%Fußnoten auf gleiche Seite
\interfootnotelinepenalty=1000

%Dateien aus Unterverzeichnissen
\usepackage{import}

%Bibliography \bibliography{literatur} und \cite{gerthsen}
%\usepackage{cite}
\usepackage{babelbib}
\selectbiblanguage{ngerman}

\begin{document}

\begin{titlepage}
\centering
\textsc{\Large Anfängerpraktikum der Fakultät für
  Physik,\\[1.5ex] Universität Göttingen}

\vspace*{4.2cm}

\rule{\textwidth}{1pt}\\[0.5cm]
{\huge \bfseries
  Fresnelsche Formeln und\\[1.5ex]
  und Polarisation}\\[0.5cm]
\rule{\textwidth}{1pt}

\vspace*{3.0cm}

\begin{Large}
\begin{tabular}{ll}
Praktikant:
 	&  Felix Kurtz\\
Versuchspartner:
 	&  Michael Lohmann\\

E-Mail: 
	&  felix.kurtz@stud.uni-goettingen.de\\
	
 Betreuer: & Phillip Bastian\\
 Versuchsdatum: &  06.03.2015\\
\end{tabular}
\end{Large}

\vspace*{1.8cm}

\begin{Large}
\fbox{
  \begin{minipage}[t][2.5cm][t]{6cm} 
    Eingegangen am:
  \end{minipage}
}
\end{Large}

\end{titlepage}

\tableofcontents

\newpage

\section{Einleitung}
\label{sec:einleitung}

\section{Theorie}
\label{sec:theorie}
\subsection{Fresnelsche Formeln}
\begin{align}
	r_s&=-\frac{\sin(\alpha-\beta)}{\sin(\alpha+\beta)}\\
	r_p&=\frac{\tan(\alpha-\beta)}{\tan(\alpha+\beta)}
\end{align}

\begin{figure}[!h]
	\centering
	\includegraphics[scale=0.7]{fresnelkoeff.png}
	\caption{Fresnelkoeffizienten für $n=1.51$. \cite[Datum: 23.03.2015]{LP20}}
	\label{fig:fresnelkoeff}
\end{figure}

\subsection{Brewster-Winkel}
\begin{align}
	\tan\alpha_\text{Brewster}=\frac{n_2}{n_1}
\end{align}

\section{Durchführung}
\label{sec:durchfuehrung}
Zuerst muss der Strahlengang justiert werden.
Dazu wird das evtl. noch im Strahlengang stehende Glasprisma entfernt und Polarisator und Analysator durchlässig gedreht.
Mit den Linsen bildet man das grüne Lichtbündel scharf auf das Okular ab.
Nun wird die Polarisationsrichtung justiert.
Dabei wird das kleine Nicolsche Prisma auf den Drehteller gestellt.
Die optische Achse des Prismas zeigt nach oben.
Man entfernt den Analysator und dreht den Polarisator, so dass im Okular kein Strahl mehr zu sehen ist.
Dann steht der Polarisator parallel zur Einfallsebene.
Jetzt wird dieser um $45^\circ$ gedreht.
Die eine Hälfte ist nun parallel, die andere senkrecht zur Einfallsebene polarisiert.
Danach wird das Glasprisma auf dem Drehteller justiert.
Meist sind Markierungen schon vorhanden.
Man prüft, ob man den Strahl durch das Okular beobachten kann, wenn der Schwenkarm nicht sowie um $90^\circ$ ausgelenkt ist.
Sollte dies nicht der Fall sein, muss nachjustiert werden.\\

Nun kann man den Reflexionskoeffizienten messen.
Dazu wird der Analysator wieder in den Strahlengang gestellt.
In $5^\circ$-Schritten wird der Schwenkarm nun von $0^\circ$ bis $90^\circ$ ausgelenkt.
Dabei dreht man den Analysator immer so, dass Dunkelheit im Okular herrscht.
Der so eingestellte Winkel wird an der Winkelskala abgelesen und notiert.\\

Zuletzt wird der Brewster-Winkel gemessen.
Dazu muss der Polarisator wieder um $45^\circ$ zurück gedreht werden, damit das Licht parallel polarisiert ist.
Der Analysator wird entfernt.
Man bestimmt mehrmals den Auslenkwinkel des Schwenkarms, bei dem ein Intensitätsminimum des reflektierten Strahls durch das Okular beobachtet wird.

\begin{figure}[!h]
	\centering
	\includegraphics[scale=0.7]{aufbau_schema.png}
	\caption{Versuchsaufbau schematisch. \cite[Datum: 23.03.2015]{LP20}}
	\label{fig:aufbau}
\end{figure}
\begin{figure}[!h]
	\centering
	\includegraphics[scale=0.7]{nicol.png}
	\caption{Strahlengang im Nicolschen Prisma. \cite[Datum: 23.03.2015]{LP20}}
	\label{fig:nicol}
\end{figure}

\section{Auswertung}
\label{sec:auswertung}
\subsection{Drehung der Schwingungsebene}
\begin{figure}[!htb]
	\centering
	\input{drehung.tex}
	\caption{Drehwinkel $\gamma$ gegen den Auftreffwinkel $\alpha$ aufgetragen.}
\end{figure}

Aus dem $\chi^2$-Fit der Theoriekurve erhält man
\begin{empheq}[box=\shadowbox]{align*}
	n = 1.405 \pm 0.019\,.
\end{empheq}

\subsection{Brechungsindex aus $\gamma=45^\circ$}
\begin{figure}[!htb]
	\centering
	\input{drehung_zoom.tex}
	\caption{$\gamma$ gegen $\alpha$ im Bereich $\gamma=45^\circ$: lineare Regression und Brewster-Winkel}
\end{figure}


\begin{align}
	\sigma_n=\frac{\sigma_\alpha}{\cos^2\alpha}
\end{align}

Aus der linearen Regression: $\alpha=53.8^\circ \pm 1.8^\circ$.
\begin{empheq}[box=\shadowbox]{align*}
	n = 1.37 \pm 0.10\,.
\end{empheq}

\subsection{Brewster-Winkel}
\begin{table}[!htb]
	\centering
	\begin{tabular}{|c|c|c|c|}
		\hline		
		& $\Phi$ [$^\circ$] & $\alpha$ [$^\circ$] & $n$ \\
		\hline
		Alle Werte & $66.6 \pm 0.6$ & $56.7 \pm 0.3$ & $1.522 \pm 0.018$ \\
		erste Messreihe & $66.4 \pm 0.9$ & $56.8 \pm 0.5$ & $1.53 \pm 0.03$ \\
		zweite Messreihe & $67.0 \pm 0.5$ & $56.50 \pm 0.25$ & $1.511 \pm 0.015$ \\
		\hline
		Michael & $67.3 \pm 0.5$ & $56.35 \pm 0.25$ & $1.502 \pm 0.015$ \\
		Felix & $65.9 \pm 1.0$ & $57.1 \pm 0.5$ & $1.55 \pm 0.03$ \\
		\hline
	\end{tabular}
	\caption{Brewster-Winkel.}
	\label{tab:brewster}
\end{table}

\section{Diskussion}
\label{sec:diskussion}

\begin{figure}[!htb]
	\centering
	\input{drehung2.tex}
	\caption{$\gamma$ gegen $\alpha$: $\gamma$ um $5^\circ$ nach oben verschoben für $\alpha \leq 67.5^\circ\,$.}
\end{figure}

\section{Anhang}

\bibliography{literatur}
\bibliographystyle{babalpha}

\end{document}
